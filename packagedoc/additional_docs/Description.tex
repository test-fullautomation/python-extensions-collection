% --------------------------------------------------------------------------------------------------------------
%
% Copyright 2020-2024 Robert Bosch GmbH

% Licensed under the Apache License, Version 2.0 (the "License");
% you may not use this file except in compliance with the License.
% You may obtain a copy of the License at

% http://www.apache.org/licenses/LICENSE-2.0

% Unless required by applicable law or agreed to in writing, software
% distributed under the License is distributed on an "AS IS" BASIS,
% WITHOUT WARRANTIES OR CONDITIONS OF ANY KIND, either express or implied.
% See the License for the specific language governing permissions and
% limitations under the License.
%
% --------------------------------------------------------------------------------------------------------------

\section{Modules}

The \pkg\ contains the following modules:

\begin{enumerate}
\item \pcode{CComparison}\\
   Compare two text files based on regular expressions.\\
   Import: \pcode{from PythonExtensionsCollection.Comparison.CComparison import CComparison}
\item \pcode{CFile}\\
   File operations like read, write, copy, move, ...\\
   Import: \pcode{from PythonExtensionsCollection.File.CFile import CFile}
\item \pcode{CFolder}\\
   Folder operations like create, delete, copy, ...\\
   Import: \pcode{from PythonExtensionsCollection.Folder.CFolder import CFolder}
\item \pcode{CString}\\
   String operations like normalize a path, string filter, format results, ...\\
   Import: \pcode{from PythonExtensionsCollection.String.CString import CString}
\item \pcode{CUtils}\\
   Pretty print of Python data types\\
   Import: \pcode{from PythonExtensionsCollection.Utils.CUtils import CTypePrint}
\end{enumerate}

\section{Methods}

Additionally to the interface descriptions in the second part of this document this section contains
a more detailed description of some assorted methods together with examples how to use.

\subsection{String operations with CString}

\subsubsection{NormalizePath}

It's not easy to handle paths - and especially the path separators - independent from the operating system.

Under Linux it is obvious that single slashes are used as separator within paths. Whereas the Windows explorer
uses single backslashes. In both operating systems web addresses contains single slashes as separator
when displayed in web browsers.

Using single backslashes within code - as content of string variables - is dangerous because the combination
of a backslash and a letter can be interpreted as escape sequence - and this is maybe not the effect a user wants to have.

To avoid unwanted escape sequences backslashes have to be masked (by the usage of two of them: \pcode{"\\\\"}).
But also this could not be the best solution because there are also applications (like the Windows explorer) that are not able to handle
masked backslashes. They expect to get single backslashes within a path.

Preparing a path for best usage within code also includes collapsing redundant separators and up-level references.
Python already provides functions to do this, but the outcome (path contains slashes or backslashes) depends on the
operating system. And like already mentioned above also under Windows backslashes might not be the preferred choice.

It also has to be considered that redundant separators at the beginning of an address of a local network resource
(like \pcode{\\\\server.com}) and or inside an internet address (like \pcode{https:\\\\server.com}) must \textbf{not} be collapsed!
Unfortunately the Python function \pcode{normpath} does not consider this context.

To give the user full control about the format of a path, independent from the operating system and independent if it's
a local path, a path to a local network resource or an internet address, the method \pcode{NormalizePath()} provides
lot's of parameters to influence the result.

\vspace{1ex}

\textbf{Example 1} (\textit{file system path})

\begin{pythoncode}
path = r"C:\\subfolder1///../subfolder2\\\\../subfolder3\\"
path = CString.NormalizePath(path)
print(path)
\end{pythoncode}

\textbf{Result} (\textit{output contains slashes})

\begin{pythonlog}
C:/subfolder3
\end{pythonlog}

\vspace{1ex}

\textbf{Example 2} (\textit{file system path})

\begin{pythoncode}
path = r"C:\\subfolder1///../subfolder2\\\\../subfolder3\\"
path = CString.NormalizePath(path, bWin=True)
print(path)
\end{pythoncode}

\textbf{Result} (\textit{output contains masked backslashes})

\begin{pythonlog}
C:\\subfolder3
\end{pythonlog}

\vspace{1ex}

\textbf{Example 3} (\textit{path of a local network resource})

\begin{pythoncode}
path = r"\\anyserver.com\\part1//part2\\\\part3/part4"
path = CString.NormalizePath(path)
print(path)
\end{pythoncode}

\textbf{Result}

\begin{pythonlog}
//anyserver.com/part1/part2/part3/part4
\end{pythonlog}

\vspace{1ex}

\textbf{Example 4} (\textit{internet address})

\begin{pythoncode}
path = r"http:\\anyserver.com\\part1//part2\\\\part3/part4"
path = CString.NormalizePath(path)
print(path)
\end{pythoncode}

\textbf{Result}

\begin{pythonlog}
http://anyserver.com/part1/part2/part3/part4
\end{pythonlog}

% --------------------------------------------------------------------------------------------------------------
\newpage

\subsubsection{DetectParentPath}

The method \pcode{DetectParentPath} computes the path to any parent folder inside a given path. Optionally \pcode{DetectParentPath} is able
to search for files inside the identified parent folder.

In the following examples we assume to have the following file system structure:

\begin{pythonlog}
D:\pathtest\ABC\DEF\GHI\FILE.txt
D:\pathtest\ABC\FILE.txt
D:\pathtest\RST\UVW\XYZ\FILE.txt
\end{pythonlog}

We are inside folder \plog{GHI} and want to know the path to folder \plog{ABC}.

\textbf{Python code}

\begin{pythoncode}
sStartPath  = r"D:\pathtest\ABC\DEF\GHI"
sFolderName = "ABC"
sDestPath, listDestPaths, sDestFile, listDestFiles, sDestPathParent = CString.DetectParentPath(sStartPath, sFolderName)
print(f"sDestPath       = {sDestPath}")
print(f"listDestPaths   = {listDestPaths}")
print(f"sDestFile       = {sDestFile}")
print(f"listDestFiles   = {listDestFiles}")
print(f"sDestPathParent = {sDestPathParent}")
\end{pythoncode}

\textbf{Outcome}

\begin{pythonlog}
sDestPath       = D:/pathtest/ABC
listDestPaths   = ['D:/pathtest/ABC']
sDestFile       = None
listDestFiles   = None
sDestPathParent = D:/pathtest
\end{pythonlog}

\pcode{sDestPath} contains the path to \pcode{sFolderName} within \pcode{sStartPath}.
\pcode{sDestPathParent} contains the path to the parent folder of \pcode{sDestPath}.
The meaning of the remaining return parameter are handled in the next examples.

\textit{It is possible to search for several folders.}

We are inside folder \plog{GHI} and want to know the path to folders \plog{ABC} and \plog{DEF}.

\begin{pythoncode}
sStartPath  = r"D:\pathtest\ABC\DEF\GHI"
sFolderName = "ABC;DEF"
sDestPath, listDestPaths, sDestFile, listDestFiles, sDestPathParent = CString.DetectParentPath(sStartPath, sFolderName)
\end{pythoncode}

\textbf{Outcome}

\begin{pythonlog}
sDestPath       = D:/pathtest/ABC/DEF
listDestPaths   = ['D:/pathtest/ABC/DEF', 'D:/pathtest/ABC']
sDestFile       = None
listDestFiles   = None
sDestPathParent = D:/pathtest/ABC
\end{pythonlog}

\pcode{sFolderName} is a semicolon separated list of folder names. Accordingly to this list of folder names \pcode{listDestPaths}
contains the paths to these folders. \pcode{DetectParentPath} searches in the path from right to left (from bottom level up to top level).
Therefore the folder \plog{DEF} is the first one who is found. The order of elements in \pcode{listDestPaths} is not synchronized
with the order of folder names in \pcode{sFolderName}!
In every case \pcode{sDestPath} contains the first element in the list.
\pcode{sDestPathParent} contains the path to the parent folder of \pcode{sDestPath}.

\textit{It is possible to search for a file.}

We are inside folder \plog{GHI} and want to know the path to folder \plog{DEF} and within this folder we want to know the path to file \plog{FILE.txt}.

\begin{pythoncode}
sStartPath  = r"D:\pathtest\ABC\DEF\GHI"
sFolderName = "DEF"
sFileName   = "FILE.txt"
sDestPath, listDestPaths, sDestFile, listDestFiles, sDestPathParent = CString.DetectParentPath(sStartPath, sFolderName, sFileName)
\end{pythoncode}

\textbf{Outcome}

\begin{pythonlog}
sDestPath       = D:/pathtest/ABC/DEF
listDestPaths   = ['D:/pathtest/ABC/DEF']
sDestFile       = D:/pathtest/ABC/DEF/GHI/FILE.txt
listDestFiles   = ['D:/pathtest/ABC/DEF/GHI/FILE.txt']
sDestPathParent = D:/pathtest/ABC
\end{pythonlog}

\pcode{listDestPaths} contains a list of all paths to \pcode{sFolderName} within \pcode{sStartPath}.
\pcode{sDestPath} contains the first element of \pcode{listDestPaths}.
\pcode{listDestFiles} contains a list of all files with name \pcode{sFileName} found within \pcode{listDestPaths}.
\pcode{sDestFile} contains the first element of \pcode{listDestFiles}.

A semicolon separated list of file names in \pcode{sFileName} (like for \pcode{sFolderName}) is not supported.

Providing more than one folder together with a file name may cause overlapping results. But \pcode{listDestFiles} will not contain
any redundant paths to files.

\begin{pythoncode}
sStartPath  = r"D:\pathtest\ABC\DEF\GHI"
sFolderName = "ABC;DEF"
sFileName   = "FILE.txt"
sDestPath, listDestPaths, sDestFile, listDestFiles, sDestPathParent = CString.DetectParentPath(sStartPath, sFolderName, sFileName)
\end{pythoncode}

\textbf{Outcome}

\begin{pythonlog}
sDestPath       = D:/pathtest/ABC/DEF
listDestPaths   = ['D:/pathtest/ABC/DEF', 'D:/pathtest/ABC']
sDestFile       = D:/pathtest/ABC/DEF/GHI/FILE.txt
listDestFiles   = ['D:/pathtest/ABC/DEF/GHI/FILE.txt', 'D:/pathtest/ABC/FILE.txt']
sDestPathParent = D:/pathtest/ABC
\end{pythonlog}

In the last example we go one further level up (\pcode{pathtest}). Because of the file search is recursive, also files in parallel trees are found now.

\begin{pythoncode}
sStartPath  = r"D:\pathtest\ABC\DEF\GHI"
sFolderName = "pathtest"
sFileName   = "FILE.txt"
sDestPath, listDestPaths, sDestFile, listDestFiles, sDestPathParent = CString.DetectParentPath(sStartPath, sFolderName, sFileName)
\end{pythoncode}

\textbf{Outcome}

\begin{pythonlog}
sDestPath       = D:/pathtest
listDestPaths   = ['D:/pathtest']
sDestFile       = D:/pathtest/ABC/DEF/GHI/FILE.txt
listDestFiles   = ['D:/pathtest/ABC/DEF/GHI/FILE.txt', 'D:/pathtest/ABC/FILE.txt', 'D:/pathtest/RST/UVW/XYZ/FILE.txt']
sDestPathParent = D:/
\end{pythonlog}

% --------------------------------------------------------------------------------------------------------------
\newpage

\subsubsection{StringFilter}

During the computation of strings there might occur the need to get to know if this string fulfils certain criteria or not.
Such a criterion can e.g. be that the string contains a certain substring. Also an inverse logic might be required:
In this case the criterion is that the string does \textbf{not} contain this substring.

It might also be required to combine several criteria to a final conclusion if in total the criterion for a string is fulfilled or not.
For example: The string must start with the string \texttt{prefix} and must also contain either the string \texttt{substring1} or
the string \texttt{substring2} but must also \textbf{not} end with the string \texttt{suffix}.

This method provides a bunch of predefined filters that can be used singly or combined to come to a final conclusion if the string fulfils all criteria or not.

These filters can be e.g. used to select or exclude lines while reading from a text file. Or they can be used to select or exclude files or folders
while walking through the file system. The filters are divided into three different types:

\begin{enumerate}
   \item Filters that are interpreted as raw strings (called 'standard filters'; no wild cards supported)
   \item Filters that are interpreted as regular expressions (called 'regular expression based filters'; the syntax of regular expressions has to be considered)
   \item Boolean switches (e.g. indicating if also an empty string is accepted or not)
\end{enumerate}

The input string might contain leading and trailing blanks and tabs. This kind of horizontal space is removed from the input string
before the standard filters start their work (except the regular expression based filters).

The regular expression based filters consider the original input string (including the leading and trailing space).

The outcome is that in case of the leading and trailing space shall be part of the criterion, the regular expression based filters can be used only.

It is possible to decide if the standard filters shall work case sensitive or not. This decision has no effect on the regular expression based filters.

The regular expression based filters always work with the original input string that is not modified in any way.

Except the regular expression based filters it is possible to provide more than one string for every standard filter (must be a semikolon separated list in this case).
A semicolon that shall be part of the search string, has to be masked in this way: \pcode{"\\;"}.

This method returns a boolean value that is \pcode{True} in case of all criteria are fulfilled, and \pcode{False} in case of some or all of them are not fulfilled.

The default value for all filters is \pcode{None} (except \pcode{bSkipBlankStrings}). In case of a filter value is \pcode{None} this filter
has no influence on the result.

In case of all filters are \pcode{None} (default) the return value is \pcode{True} (except the string itself is \pcode{None}
or the string is empty and \pcode{bSkipBlankStrings} is \pcode{True}).

In case of the string is \pcode{None}, the return value is \pcode{False}, because nothing concrete can be done with \pcode{None} strings.

Internally every filter has his own individual acknowledge that indicates if the criterion of this filter is fulfilled or not.

The meaning of \textit{criterion fulfilled} of a filter is that the filter supports the final return value \pcode{bAck} of this method with \pcode{True}.

The final return value \pcode{bAck} of this method is a logical join (\texttt{AND}) of all individual acknowledges (except \pcode{bSkipBlankStrings}
and \pcode{sComment}; in case of their criteria are fulfilled, immediately \pcode{False} is returned).

\vspace{1ex}

Summarized:

\begin{itemize}
   \item Filters are used to define \textit{criteria}
   \item The return value of this method provides the \textit{conclusion} - indicating if all criteria are fulfilled or not
\end{itemize}

All available filters are described in more detail in the interface description of \pcode{StringFilter}. Here we continue with some code examples.

% --------------------------------------------------------------------------------------------------------------
\newpage

\textbf{Example 1}

\pcode{sString} has to start with \pcode{sStartsWith} and has to contain \pcode{sContains}.

That's true. Therefore \pcode{StringFilter} returns \pcode{True}.

\begin{pythoncode}
StringFilter(sString           = "Speed is 25 beats per minute",
             bCaseSensitive    = True,
             bSkipBlankStrings = True,
             sComment          = None,
             sStartsWith       = "Sp",
             sEndsWith         = None,
             sStartsNotWith    = None,
             sEndsNotWith      = None,
             sContains         = "beats",
             sContainsNot      = None,
             sInclRegEx        = None,
             sExclRegEx        = None)
\end{pythoncode}

\textbf{Example 2}

\pcode{sString} must not end with \pcode{sEndsNotWith}. But does.
Therefore \pcode{StringFilter} returns \pcode{False} - even in case of other criterions are fulfilled.

\begin{pythoncode}
StringFilter(sString           = "Speed is 25 beats per minute",
             bCaseSensitive    = True,
             bSkipBlankStrings = True,
             sComment          = None,
             sStartsWith       = "Sp",
             sEndsWith         = None,
             sStartsNotWith    = None,
             sEndsNotWith      = "minute",
             sContains         = "beats",
             sContainsNot      = None,
             sInclRegEx        = None,
             sExclRegEx        = None)
\end{pythoncode}

\textbf{Example 3}

\pcode{sString} must not contain \pcode{sContainsNot}. Because \pcode{bCaseSensitive} is \pcode{True} the spelling does not fit.
Therefore \pcode{StringFilter} returns \pcode{True}.

\begin{pythoncode}
StringFilter(sString           = "Speed is 25 beats per minute",
             bCaseSensitive    = True,
             bSkipBlankStrings = True,
             sComment          = None,
             sStartsWith       = None,
             sEndsWith         = None,
             sStartsNotWith    = None,
             sEndsNotWith      = None,
             sContains         = None,
             sContainsNot      = "Beats",
             sInclRegEx        = None,
             sExclRegEx        = None)
\end{pythoncode}

% --------------------------------------------------------------------------------------------------------------
\newpage

\textbf{Example 4}

\pcode{sString} must contain exactly two digits (postulated by regular expression based \pcode{sInclRegEx}).
That's true. Therefore \pcode{StringFilter} returns \pcode{True}.

\begin{pythoncode}
StringFilter(sString           = "Speed is 25 beats per minute",
             bCaseSensitive    = True,
             bSkipBlankStrings = True,
             sComment          = None,
             sStartsWith       = None,
             sEndsWith         = None,
             sStartsNotWith    = None,
             sEndsNotWith      = None,
             sContains         = None,
             sContainsNot      = None,
             sInclRegEx        = r"\d{2}",
             sExclRegEx        = None)
\end{pythoncode}

\textbf{Example 5}

\pcode{sString} must contain exactly three digits (postulated by regular expression based \pcode{sInclRegEx}).
That's not true. Therefore \pcode{StringFilter} returns \pcode{False} - even in case of other criterions are fulfilled.

\begin{pythoncode}
StringFilter(sString           = "Speed is 25 beats per minute",
             bCaseSensitive    = True,
             bSkipBlankStrings = True,
             sComment          = None,
             sStartsWith       = "Speed",
             sEndsWith         = None,
             sStartsNotWith    = None,
             sEndsNotWith      = None,
             sContains         = None,
             sContainsNot      = None,
             sInclRegEx        = r"\d{3}",
             sExclRegEx        = None)
\end{pythoncode}

\textbf{Example 6}

Leading and trailing spaces are removed from the input string \pcode{sString} at the beginning. In this example the result is an empty input string.
\pcode{bSkipBlankStrings} is set to \pcode{True}. In this case \pcode{StringFilter} immediately returns \pcode{False}
and all other filters are ignored.

\begin{pythoncode}
StringFilter(sString           = "     ",
             bCaseSensitive    = True,
             bSkipBlankStrings = True,
             sComment          = None,
             sStartsWith       = None,
             sEndsWith         = None,
             sStartsNotWith    = None,
             sEndsNotWith      = None,
             sContains         = None,
             sContainsNot      = None,
             sInclRegEx        = None,
             sExclRegEx        = None)
\end{pythoncode}

% --------------------------------------------------------------------------------------------------------------
\newpage

\textbf{Example 7}

The input string \pcode{sString} starts with a character that is defined to be a comment character (\pcode{sComment}).
Therefore \pcode{StringFilter} immediately returns \pcode{False} and all other filters are ignored.

\begin{pythoncode}
StringFilter(sString           = "# Speed is 25 beats per minute",
             bCaseSensitive    = True,
             bSkipBlankStrings = True,
             sComment          = "#",
             sStartsWith       = None,
             sEndsWith         = None,
             sStartsNotWith    = None,
             sEndsNotWith      = None,
             sContains         = "beats",
             sContainsNot      = None,
             sInclRegEx        = None,
             sExclRegEx        = None)
\end{pythoncode}

\textbf{Example 8}

Blanks around search strings (here \pcode{sContains} is \pcode{"\ \ \ Alpha "}) are considered, whereas the blanks around the input string are removed before computation.
Therefore \pcode{"\ \ \ Alpha "} cannot be found within the (shortened) input string and \pcode{StringFilter} returns \pcode{False}.

\begin{pythoncode}
StringFilter(sString           = "   Alpha is not beta; and beta is not gamma  ",
             bCaseSensitive    = True,
             bSkipBlankStrings = True,
             sComment          = None,
             sStartsWith       = None,
             sEndsWith         = None,
             sStartsNotWith    = None,
             sEndsNotWith      = None,
             sContains         = "   Alpha ",
             sContainsNot      = None,
             sInclRegEx        = None,
             sExclRegEx        = None)
\end{pythoncode}

\textbf{Example 9}

In case of blanks around search strings have to be considered, a regular expression based filter has to be used (here \pcode{sInclRegEx}).

This is possible because the regular expression based filters \pcode{sInclRegEx} and \pcode{sExclRegEx} work \textbf{with the original value}
of \pcode{sString}, and not with the shortened version with leading and trailing blanks removed!
The shortened version is applied to standard filters only.

In this example \pcode{StringFilter} returns \pcode{True}.

\begin{pythoncode}
StringFilter(sString           = "   Alpha is not beta; and beta is not gamma  ",
             bCaseSensitive    = True,
             bSkipBlankStrings = True,
             sComment          = None,
             sStartsWith       = None,
             sEndsWith         = None,
             sStartsNotWith    = None,
             sEndsNotWith      = None,
             sContains         = None,
             sContainsNot      = None,
             sInclRegEx        = r"\s{3}Alpha",
             sExclRegEx        = None)
\end{pythoncode}

% --------------------------------------------------------------------------------------------------------------
\newpage

\textbf{Example 10}

The meaning of \pcode{"beta; and"} in this example is: The criterion is fulfilled in case of either \pcode{"beta"} or \pcode{" and"} can be found.
That's true - but this has nothing to do with the fact, that also this string \pcode{"beta; and"} can be found. The semikolon is a separator
character and therefore part of the syntax.

Nevertheless \pcode{StringFilter} returns \pcode{True}.

\begin{pythoncode}
StringFilter(sString           = "Alpha is not beta; and beta is not gamma",
             bCaseSensitive    = True,
             bSkipBlankStrings = True,
             sComment          = None,
             sStartsWith       = None,
             sEndsWith         = None,
             sStartsNotWith    = None,
             sEndsNotWith      = None,
             sContains         = "beta; and",
             sContainsNot      = None,
             sInclRegEx        = None,
             sExclRegEx        = None)
\end{pythoncode}

\textbf{Example 11}

In this example the semicolon is masked with \pcode{"\\;"} and therefore part of the search string \pcode{sContains} - and
not part of the syntax any more.

The meaning of \pcode{"beta\\; not"} in this example is: The criterion is fulfilled in case of \pcode{"beta; not"} can be found.
That's \textbf{not} \pcode{True}. Therefore \pcode{StringFilter} returns \pcode{False}.

\begin{pythoncode}
StringFilter(sString           = "Alpha is not beta; and beta is not gamma",
             bCaseSensitive    = True,
             bSkipBlankStrings = True,
             sComment          = None,
             sStartsWith       = None,
             sEndsWith         = None,
             sStartsNotWith    = None,
             sEndsNotWith      = None,
             sContains         = r"beta\; not",
             sContainsNot      = None,
             sInclRegEx        = None,
             sExclRegEx        = None)
\end{pythoncode}

% --------------------------------------------------------------------------------------------------------------
\newpage

\subsection{File access with CFile}

The motivation for the \pcode{CFile} module contains two main topics:

\begin{enumerate}
   \item Extended user control by introducing further parameter for file access functions. With high priority \pcode{CFile} enables the user
         to take care about that nothing existing is overwritten accidently.
   \item Hide the file handles und use the mechanism of class variables to avoid access violations independent from
         the way different operation systems like Windows and Unix are handling this.
\end{enumerate}

This shortens the code, eases the implementation and makes tests (in which this module is used) more stable.

\vspace{1ex}

\textbf{Examples}

\textit{Define two variables with path and name of test files.}

Under Windows:

\begin{pythoncode}
sFile_1 = r"%TMP%\CFile_TestFile_1.txt"
sFile_2 = r"%TMP%\CFile_TestFile_2.txt"
\end{pythoncode}

Or under Linux:

\begin{pythoncode}
sFile_1 = r"/tmp/CFile_TestFile_1.txt"
sFile_2 = r"/tmp/CFile_TestFile_2.txt"
\end{pythoncode}

\textit{The first class instance:}

\begin{pythoncode}
oFile_1 = CFile(sFile_1)
\end{pythoncode}

\pcode{oFile_1} is the instance of a class - \textit{and not the file handle}. The file handle is hidden, the user has nothing to do with it.

Every class instance can work with one single file only (during the complete instance lifetime) and has exclusive access to this file.

No other class instance is allowed to use this file. Therefore the second line in the following code throws an exception:

\begin{pythoncode}[linebackgroundcolor=\hlcode{2}]
oFile_1_A = CFile(sFile_1)
oFile_1_B = CFile(sFile_1)
\end{pythoncode}

It's more save to implement in this way:

\begin{pythoncode}
try:
   oFile_1 = CFile(sFile_1)
except Exception as reason:
   print(str(reason))
\end{pythoncode}

For writing content to files two methods are available: \pcode{Write()} and \pcode{Append()}.

\begin{itemize}
   \item Using \pcode{Write()} causes the class to open the file for writing (\pcode{w}) - in case of the file is not already opened for writing.
   \item Using \pcode{Append()} causes the class to open the file for appending (\pcode{a}) - in case of the file is not already opened for appending.
\end{itemize}

Switching between \pcode{Write()} and \pcode{Append()} causes an intermediate file handle \pcode{close()} internally!

\textit{Write some content to file:}

\begin{pythoncode}
bSuccess, sResult = oFile_1.Write("A B C")
print(f"sResult oFile_1.Write : '{sResult}' / bSuccess : {bSuccess}")
\end{pythoncode}

Most of the functions return at least \pcode{bSuccess} and \pcode{sResult}.

\begin{itemize}
   \item \pcode{bSuccess} is \pcode{True} in case of no error occurred.
   \item \pcode{bSuccess} is \pcode{False} in case of an error occurred.
   \item \pcode{bSuccess} is \pcode{None} in case of a very fatal error occurred - like an exception.
   \item \pcode{sResult} contains details about what happens during computation.
\end{itemize}

It is possible now to continue with using \pcode{oFile_1.Write("...")}; the content will be appended - as long as the file
is still open for writing.

Some functions close the file handle (e.g. \pcode{ReadLines()}). Therefore sequences like

\begin{pythoncode}
oFile_1.Write("...")
oFile_1.ReadLines("...")
oFile_1.Write("...")
\end{pythoncode}

should be avoided - because \pcode{Write()} after \pcode{ReadLines()} starts the file from scratch and the file content
written by the previous \pcode{Write()} calls is lost.

For appending content to a file use the function \pcode{Append()}.

\textit{Append content to file:}

\begin{pythoncode}
bSuccess, sResult = oFile_1.Append("A B C")
\end{pythoncode}

For reading content from a file use the function \pcode{ReadLines()}.

\textit{Read from file:}

\begin{pythoncode}
listLines_1, bSuccess, sResult = oFile_1.ReadLines()
for sLine in listLines_1:
   print(f"{sLine}")
\end{pythoncode}

Additionally to \pcode{bSuccess} and \pcode{sResult} the function returnes a list of lines.

Internally \pcode{ReadLines()} takes care about:

\begin{itemize}
   \item Closing the file - in case the file is still opened
   \item Opening the file for reading
   \item Reading the content line by line until the end of file is reached
   \item Closing the file
\end{itemize}

To avoid code like this

\begin{pythoncode}
for sLine in listLines_1:
   print(f"{sLine}")
\end{pythoncode}

it is also possible to let \pcode{ReadLines()} do this:

\begin{pythoncode}
listLines_1, bSuccess, sResult = oFile_1.ReadLines(bToScreen=True)
\end{pythoncode}

A function to read only a single line from a file is not available, but it is possible to use some filter parameter of \pcode{ReadLines()}
to reduce the amount of content already during the file is read. This prevents the user from implementing further loops.

Internally \pcode{ReadLines()} uses the string filter method \pcode{StringFilter()}. All filter related input parameter of
\pcode{ReadLines()} and \pcode{StringFilter()} are the same.

Let's assume the following:

\begin{itemize}
   \item The file \pcode{sFile_1} contains empty lines
   \item The file \pcode{sFile_1} contains also lines, that are commented out (with a hash (\pcode{\#}) at the beginning)
   \item We want \pcode{ReadLines()} to skip empty lines and lines that are commented out
\end{itemize}

This can be imlemented in the following way.

\textit{Read a subset of file content:}

\begin{pythoncode}
listLines_1, bSuccess, sResult = oFile_1.ReadLines(bSkipBlankLines=True,
                                                   sComment='#')
\end{pythoncode}

It is a good practice to close file handles as soon as possible. Therefore \pcode{CFile} provides the possibility to do this explicitely.

\textit{Close a file handle:}

\begin{pythoncode}
bSuccess, sResult = oFile_1.Close()
\end{pythoncode}

This makes sense in case of later again access to this file is needed (the class object \pcode{oFile_1} still exists).

Additionally to that the file handle is closed implicitely:

\begin{itemize}
   \item in case of it is required (e.g. when switching between read and write access),
   \item in case of the class instance is destoyed.
\end{itemize}

Therefore an alternative to the \pcode{Close()} function is the deletion of the class instance:

\begin{pythoncode}
del oFile_1
\end{pythoncode}

This makes sense in case of access to this file is not needed any more (therefore we also do not need the class object any more).

It is recommended to prefer \pcode{del} (instead of \pcode{Close()}) to avoid to keep too much not used objects for a too long length of time in memory.

A file can be copied to another file.

\textit{Copy a file:}

\begin{pythoncode}
bSuccess, sResult = oFile_1.CopyTo(sFile_2)
\end{pythoncode}

The destination (\pcode{sFile_2} in the example above) can either be a full path and name of a file or the path only.

It makes a difference if the destination file exists or not. The optional parameter \pcode{bOverwrite} controls the behavior of \pcode{CopyTo()}.

The default is that it is not allowed to overwrite an existing destination file: \pcode{bOverwrite} is \pcode{False}. \pcode{CopyTo()} returns
\pcode{bSuccess = False} in this case.

In case the user want to allow \pcode{CopyTo()} to overwrite existing destination files, it has to be coded explicitely:

\begin{pythoncode}
bSuccess, sResult = oFile_1.CopyTo(sFile_2, bOverwrite=True)
\end{pythoncode}

A file can be moved to another file.

\textit{Move a file:}

\begin{pythoncode}
bSuccess, sResult = oFile_1.MoveTo(sFile_2)
\end{pythoncode}

Also \pcode{MoveTo()} supports \pcode{bOverwrite}. The behavior is the same as \pcode{CopyTo()}.

A file can be deleted.

\textit{Delete a file:}

\begin{pythoncode}
bSuccess, sResult = oFile_1.Delete()
\end{pythoncode}

It is possible to distinguish between two different motivations to delete a file:

\begin{enumerate}
   \item \textbf{Explicitely do a deletion}\\
         This requires that the file to be deleted, does exist.
   \item \textbf{Making sure only that the files does not exist}\\
         In this case it doesn't matter that maybe there is nothing to delete because the file already does not exist.
\end{enumerate}

The optional parameter \pcode{bConfirmDelete} controls this behavior.

Default is that \pcode{Delete()} requires an existing file to delete:

\begin{pythoncode}
bSuccess, sResult = oFile_1.Delete(bConfirmDelete=True)
\end{pythoncode}

In case of the file does not exist, \pcode{Delete()} returns \pcode{bSuccess = False}.

\pcode{Delete()} also returns \pcode{bSuccess = False|None} in case of an existing file cannot be deleted (e.g. because of an access violation).

If it doesn't matter it the file exists or not, it has to be coded explicitely:

\begin{pythoncode}
bSuccess, sResult = oFile_1.Delete(bConfirmDelete=False)
\end{pythoncode}

In this case \pcode{Delete()} only returns \pcode{bSuccess = False|None} in case of an existing file cannot be deleted (e.g. because of an access violation).

\textbf{Avoid access violations}

Like already mentioned above every instance of \pcode{CFile} has an exclusive access to it's own file.

Only in case of \pcode{CopyTo()} and \pcode{MoveTo()} other files are involved: the destination files.

To avoid access violations it is not possible to copy or move a file to another file, that is under access of another instance of \pcode{CFile}.

In the following example \pcode{oFile_1.CopyTo(sFile_2)} returns \pcode{bSuccess = False} because \pcode{sFile_2} is already in access by \pcode{oFile_2}.

\begin{pythoncode}[linebackgroundcolor=\hlcode{7}]
oFile_1 = CFile(sFile_1)
bSuccess, sResult = oFile_1.Write("A B C")

oFile_2 = CFile(sFile_2)
listLines_2, bSuccess, sResult = oFile_2.ReadLines()

bSuccess, sResult = oFile_1.CopyTo(sFile_2)

del oFile_1
del oFile_2
\end{pythoncode}

The solution is to delete the class instances as early as possible.

In the following example the copying is successful:

\begin{pythoncode}[linebackgroundcolor=\hlcode{6}]
oFile_1 = CFile(sFile_1)
bSuccess, sResult = oFile_1.Write("A B C")

oFile_2 = CFile(sFile_2)
listLines_2, bSuccess, sResult = oFile_2.ReadLines()
del oFile_2

bSuccess, sResult = oFile_1.CopyTo(sFile_2)
del oFile_1
\end{pythoncode}

% --------------------------------------------------------------------------------------------------------------
\newpage

\subsection{Utilities}

\subsubsection{PrettyPrint}

The idea behind the \pcode{PrettyPrint()} function is to resolve the content of composite data types and provide for every parameter inside:

\begin{itemize}
   \item the type
   \item the total number of elements inside (e.g. the number of keys inside a dictionary)
   \item the counter number of the current element
   \item the value
\end{itemize}

\textbf{Example}

The following Python code defines a composite data type and prints the content with \pcode{PrettyPrint()}:

\begin{pythoncode}
dictTest = {'K1' : 'value',
            'K2' : ["A", 22, True, (33, 'XYZ')],
            'K3' : 10,
            'K4' : {'A' : 1,
                    'B' : 2}}

PrettyPrint(dictTest)
\end{pythoncode}

\textbf{Result}

\begin{pythonlog}
[DICT] (4/1) > {K1} [STR]  :  'value'
[DICT] (4/2) > {K2} [LIST] (4/1) > [STR]  :  'A'
[DICT] (4/2) > {K2} [LIST] (4/2) > [INT]  :  22
[DICT] (4/2) > {K2} [LIST] (4/3) > [BOOL]  :  True
[DICT] (4/2) > {K2} [LIST] (4/4) > [TUPLE] (2/1) > [INT]  :  33
[DICT] (4/2) > {K2} [LIST] (4/4) > [TUPLE] (2/2) > [STR]  :  'XYZ'
[DICT] (4/3) > {K3} [INT]  :  10
[DICT] (4/4) > {K4} [DICT] (2/1) > {A} [INT]  :  1
[DICT] (4/4) > {K4} [DICT] (2/2) > {B} [INT]  :  2
\end{pythonlog}

Every line of output has to be interpreted strictly from left to right.

For example the meaning of the fifth line of output

\begin{pythonlog}
[DICT] (4/2) > {K2} [LIST] (4/4) > [TUPLE] (2/1) > [INT]  :  33
\end{pythonlog}

is:

\begin{itemize}
   \item The type of input parameter \plog{oData} is \plog{dict}
   \item The dictionary contains 4 keys
   \item The current line gives information about the second key of the dictionary
   \item The name of the second key is \plog{K2}
   \item The value of the second key is of type \plog{list}
   \item The list contains 4 elements
   \item The current line gives information about the fourth element of the list
   \item The fourth element of the list is of type \plog{tuple}
   \item The tuple contains 2 elements
   \item The current line gives information about the first element of the tuple
   \item The first element of the tuple is of type \plog{int} and has the value \plog{33}
\end{itemize}

Types are encapsulated in square brackets, counter in round brackets and key names are encapsulated in curly brackets.

